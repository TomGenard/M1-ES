\documentclass{article}

\title{Rapport de Traveaux Pratiques\\
  Exp\'{e}riances et Simulations\\
  S\'{e}quance 1}

\author{G\'{E}NARD Tom \\
  Master de Physique, niveau M1\\
  Universit\'{e} de Caen Normandie\\
  Contact: \url{mailto:21502237@etu.unicaen.fr}}
  
\date{\today}

\begin{document}

\maketitle
\vfill
\pagebreak

\tableofcontents
\vfill
\pagebreak

\section{Introduction}\label{section:intro}
Dans cette premi\`{e}re s\'{e}quance, nous allons tout d'abord nous intéresser à des générateurs simples de nombre aléatoire, et nous allons nous re-familiariser avec Python3.

\section{Atelier 1}\label{section:atelier1}
Nous commençons tout d'abord par créer le générateur de nombres aléatoire que nous allons principalement utiliser. Ce générateur est un générateur pseudo-aléatoire basé sur un algorithme dit de congruance linéaire, et plus particulièrement celui de Knuth et Lewis.\\
\newline
L'algorithme de Knuth et Lewis permet de générer des valeurs semblant aléatoire à l'aide de la formule congruance linéaire suivante :
\begin{equation}
  r_{n+1} = ( r_{n} * a + c ) \; mod \; 2^{m} \label{eq:ann:1}
\end{equation}
avec pour l'algorithme de Knuth et Lewis :
\begin{equation}
  a = 1664525
\end{equation}
\begin{equation}
  c = 1013904223
\end{equation}
\begin{equation}
  m = 32
\end{equation}

\end{document}